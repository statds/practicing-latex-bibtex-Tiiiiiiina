\documentclass[12pt]{article}

%% preamble: Keep it clean; only include those you need
\usepackage{amsmath}
\usepackage[margin = 1in]{geometry}
\usepackage{graphicx}
\usepackage{booktabs}
\usepackage{natbib}

% for space filling
\usepackage{lipsum}
% highlighting hyper links
\usepackage[colorlinks=true, citecolor=blue]{hyperref}


%% meta data

\title{Impact of diet on chronic kidney disease}
\author{SHIYI PENG\\
  Department of Statistics\\
  University of Connecticut
}

\begin{document}
\maketitle

\begin{abstract}

\citet{Van2020Diet}Healthy dietary patterns are associated with a lower risk of CKD. More research is needed into the effects of specific food groups and beverages on kidney function.\lipsum[1]

\end{abstract}


\section{Introduction}
\label{sec:intro}

Chronic Kidney Disease (CKD) is a global health burden that increases the risk of premature death or reduced quality of life. However, CKD does not present with obvious symptoms until the later stages of the disease. Therefore,It is easy to ignore CKD until it becomes end-stage renal disease, which requires dialysis or kidney transplant for whole life\citet{Hill2016Global}. We need to know how to protect our kidneys in our daily lives and how to slow down the progression of chronic kidney disease (CKD). Carbonated drinks and sugar-free beverages are now popular among young people, so it is important to know whether such drinks are harmful to the kidneys. Also everyone should consume a certain amount of carbohydrates every day, however few studies have investigated the effect of carbohydrate intake on CKD patients\citet{Ren2023Associations},so another part of the article will be centered on carbohydrate intake.

% roadmap
The rest of the paper is organized as follows.
The data will be presented in Section~\ref{sec:data}.
The methods are described in Section~\ref{sec:meth}.
The results are reported in Section~\ref{sec:resu}.
A discussion concludes in Section~\ref{sec:disc}.


\section{Data}
\label{sec:data}
\begin{equation}
  IQR=Q3-Q1
\end{equation}
$IQR$ is the interquartile range contains the second and third quartiles, or the middle half of your data set. 

\section{Methods}
\label{sec:meth}

Use this section to present the methodologies that will generate results by
analyzing the data. Suppose that in eGFR 0-59 mL/min·1.73m2 group, the mortality risk was lower in participants who had 30-45% energy from carbohydrate (average HR 0.68, 95%CI 0.51-0.90, compared with 60%), 5%-20% energy from sugar (average HR 0.64, 95%CI 0.48-0.87, compared with 40%), 40-50% energy from non-sugar carbohydrate (average HR 0.58, 95%CI 0.39-0.85, compared with 30%)\citet{Hill2016Global}. 

\begin{equation}
eGFR = 175 x (SCr)^(-1.154) x(age)^(-0.203) xSex[male=1,female=1.75] x1.212 [if Black]
\end{equation}
which states that $eGFR$ is estimated glomerular filtration rate (mL/min/1.73 m^2),$Scr$ is standardized serum creatinine( mg/dL),$age$ is years.

\section{Results}
\label{sec:resu}
c
Table~\ref{tab:rv} summarizes some example draws from some distributions.
\lipsum[1-4]

\begin{table}[tbp]
  \caption{This is my first table.}
  \label{tab:rv}
\centering
\begin{tabular}{rrr}
  \toprule
normal & poisson & gamma \\ 
  \midrule
-0.110 & 4 & 2.401 \\ 
  0.116 & 4 & 3.529 \\ 
  -0.828 & 9 & 2.112 \\ 
  -0.066 & 6 & 11.104 \\ 
  0.219 & 3 & 4.815 \\ 
  0.303 & 5 & 2.188 \\ 
  0.544 & 0 & 8.050 \\ 
  -2.617 & 8 & 3.646 \\ 
  0.747 & 1 & 5.178 \\ 
  -1.103 & 4 & 3.043 \\ 
   \bottomrule
\end{tabular}
\end{table}

Figure~\ref{fig:cars} shows the distance against the speed from this dataset.


\begin{figure}[tbp]
  \centering
  \includegraphics[width=\textwidth]{cars.pdf}
  \caption{This is my first figure.}
  \label{fig:cars}
\end{figure}

\section{Discussion}
\label{sec:disc}

What are the main contributions again?

What are the limitations of this study?

What are worth pursuing further in the future?

\lipsum[1-2]
Watch for prevalence of diabetes \citep{wild2004global}.

\bibliography{refs}
\bibliographystyle{mcap}

\end{document}
