\documentclass{article}
\usepackage{graphicx} % Required for inserting images
\usepackage{algorithm}
\usepackage{color}
\usepackage{amssymb}
\usepackage{setspace}
\usepackage{parskip}
\usepackage{listings}
\usepackage{amsmath}
\usepackage{natbib}

\textwidth 5.5in
\textheight 9 in
\topmargin -0.5in
\oddsidemargin 0.5in
\evensidemargin 0.00in
\onehalfspacing
\usepackage{indentfirst}
\setlength{\parindent}{1cm}

\newcommand{\greet}[0]{$\lambda$}

\begin{document}
\begin{center}
    \textbf{\large A Clear View on Lasso}\\
    {Lai Jiang} \\ Department of Statistics, University of Connecticut
\end{center}

\begin{abstract}
    Lasso, short for "Least Absolute Shrinkage and Selection Operator," is a linear regression technique used in statistics and machine learning. It is a regularization method that helps prevent over fitting and can be used for both regression and feature selection tasks. Lasso works by adding a penalty term to the linear regression equation that encourages the model to shrink the coefficients of less important features to zero, effectively removing them from the model. This makes Lasso particularly useful when dealing with high-dimensional data-sets with many features, where feature selection and simplification are important.\\
    %\greet{Bluewise}
\end{abstract}

\section*{Introduction}
Lasso is a widely used regularization technique in statistics and machine learning, particularly in the context of linear regression. It was introduced to address common issues encountered in regression analysis, such as multicollinearity and overfitting, by adding a penalty term to the standard linear regression objective function.

The core idea behind Lasso is to encourage the model to not only minimize the sum of squared residuals (the difference between predicted and actual values) but also to minimize the sum of the absolute values of the regression coefficients. This additional penalty term is often represented by the L1 norm of the coefficient vector. Mathematically, Lasso can be expressed as:\\\begin{equation}
\min_{\beta} \left\{ \frac{1}{2n} \sum_{i=1}^{n} (y_i - X_i^T \beta)^2 + \lambda \|\beta\|_1 \right\}
\end{equation}

What makes Lasso distinctive is its ability to perform feature selection. As the value of \greet increases, more coefficients in the model are shrunk towards zero, effectively setting them to zero when the penalty becomes sufficiently strong. This means that Lasso can automatically identify and discard less relevant features from the model, thus simplifying the model and reducing the risk of over fitting. In practice, choosing the appropriate value of $\lambda$ is crucial. A small $\lambda$ may lead to a model similar to standard linear regression, while a large $\lambda$ may lead to a model with most coefficients set to zero. Cross-validation techniques are often used to determine the optimal $\lambda$ that balances model simplicity with predictive performance.

In this article I'll introduce the Table and Figure in Section 2, the in-display Mathematical Formula in Section 3, References in Section 4, Appendix in Section 5, and another citation in a separate file.

\section*{Table and Figure}
The following contents will be out of the contents of Lasso since more research is needed. In this section, I will provide the example of table and figure. The Data (Table) describes the basic information of some random Superheros; the data was also be used in the first homework of STAT 3675Q. Here is a Visual Version of the data, and you may also find the data file under the Code Folder:

\begin{table}[h]
\begin{center}
\caption{Superheros}
\begin{tabular}{|l|l|l|l|l|}
\hline
Name      & Sex & Age & Superhero & Tattos \\ \hline
Astrid    & F   & 30  & Batman    & 11     \\ \
Lea       & F   & 25  & Superman  & 15     \\ \
Sarina    & F   & 25  & Batman    & 12     \\ \
Remon     & M   & 29  & Spiderman & 5      \\ \
Letizia   & F   & 22  & Batman    & 65     \\ \
Babice    & F   & 22  & Antman    & 3      \\ \
Jonas     & M   & 35  & Batman    & 9      \\ \
Wendy     & F   & 19  & Superman  & 13     \\ \
Niveditha & F   & 32  & Maggott   & 900    \\ \
Gioia     & F   & 21  & Superman  & 0      \\ \hline
\end{tabular}
\end{center}
\end{table}
And Here is a plot of the distribution of Age:
\begin{figure}[htbp]
    \centering
    \includegraphics[width=0.8\textwidth]{Figure.pdf}
    \caption{Your Figure Caption}
    \label{fig:your-label}
\end{figure}
\section*{In-Display Mathematical Formula}
The following contents are just picked from ChatGPT to make this article similar to a paper. The mathematical Formula located below: Statistics is a branch of mathematics and a fundamental tool in the realm of data analysis and decision-making. It plays a crucial role in various fields, including science, business, social sciences, and beyond. At its core, statistics is concerned with collecting, organizing, analyzing, interpreting, and presenting data to gain insights and make informed decisions.

One of the primary objectives of statistics is to summarize and describe data. Descriptive statistics involve techniques for summarizing data, such as calculating measures of central tendency (e.g., mean, median, mode), measures of dispersion (e.g., variance, standard deviation, range), and graphical representations (e.g., histograms, box plots) that provide a clear overview of the data's characteristics.

Inferential statistics, on the other hand, is about making inferences or predictions about a population based on a sample of data. (Albert Einstein)It involves hypothesis testing, confidence intervals, and regression analysis, among other techniques. Inferential statistics allows us to draw conclusions and make decisions even when we don't have access to the entire population.

Statistical methods are used in a wide range of applications. In scientific research, statistics help researchers test hypotheses, draw meaningful conclusions, and identify patterns or relationships in data. In business and economics, statistics aid in market research, forecasting, quality control, and risk assessment. In healthcare, statistics are crucial for clinical trials, epidemiological studies, and medical research.\citep{knuth1984}

With the advent of technology, data collection and analysis have become more accessible and powerful. Data science, a multidisciplinary field that combines statistics, computer science, and domain expertise, leverages statistical techniques to extract valuable insights from large and complex datasets. Machine learning, a subset of data science, relies heavily on statistical algorithms for tasks like classification, regression, and clustering.

Suppose we want to find the PDF of a random variable Z, which is defined as:
    \begin{equation}
        Z=X^2+Y^2
    \end{equation}
    
How should we find the PDF of this function?\\\\
1. Joint PDF of X and Y:
\begin{equation}
f_{XY}(x, y) = \frac{1}{2\pi\sigma_X\sigma_Y} \exp\left(-\frac{(x-\mu_X)^2}{2\sigma_X^2} - \frac{(y-\mu_Y)^2}{2\sigma_Y^2}\right)
\end{equation}
2. Transformation of Variables:
\begin{align}
x &= \sqrt{z - y^2} \\
y &= \sqrt{z - x^2}
\end{align}
3. Jacobian Transformation:
\begin{equation}
\left|Jacobian\right| = 4xy
\end{equation}
4. PDF of Z:
\begin{align}
f_Z(z) &= f_{XY}(x, y) \cdot \left|Jacobian\right| \\
&= \frac{1}{2\pi\sigma_X\sigma_Y} \exp\left(-\frac{(\sqrt{z - y^2}-\mu_X)^2}{2\sigma_X^2} - \frac{(\sqrt{z - x^2}-\mu_Y)^2}{2\sigma_Y^2}\right) \cdot 4xy
\end{align}

The importance of statistics cannot be overstated in our data-driven world. It empowers individuals and organizations to make data-driven decisions, solve problems, and uncover hidden patterns and trends. Whether you're a scientist, business analyst, policymaker, or simply a curious individual, understanding and applying statistical principles can be a valuable skill for navigating an increasingly data-rich environment.\citep{einstein1905}


\section*{Appendix}
This is the Appendix Section, more contents will be here in Final Draft.

\bibliographystyle{plain}
\bibliography{Citations.bib}
\cite{knuth1984}
\cite{einstein1905}

\end{document}
